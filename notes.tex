\documentclass[10pt]{article}
\usepackage{graphicx}
\usepackage{parskip}
\usepackage{amsmath}
\usepackage{float}
\usepackage{listings}
\usepackage{color}
\usepackage{ulem}
\usepackage[margin = 1in]{geometry}

\definecolor{mygreen}{rgb}{0,0.6,0}
\definecolor{mygray}{rgb}{0.5,0.5,0.5}
\definecolor{mymauve}{rgb}{0.58,0,0.82}

\lstset{ %
  backgroundcolor=\color{white},   % choose the background color
  basicstyle=\footnotesize,        % size of fonts used for the code
  breaklines=true,                 % automatic line breaking only at whitespace
  captionpos=b,                    % sets the caption-position to bottom
  commentstyle=\color{mygreen},    % comment style
  escapeinside={\%*}{*)},          % if you want to add LaTeX within your code
  keywordstyle=\color{blue},       % keyword style
  stringstyle=\color{mymauve},     % string literal style
}

\begin{document}

\title{Progress Notes on Motor Design Code}
\author{Garo Bedonian, Aaron Grgurich}

\maketitle

\section{NEXT STEP}

\sout{Talk with Aaron about MotorCAD manual, talk about FEM with Justin,} use total mass as objective

\sout{Use Tucker's resources to figure out what to do to model loss}

Hanselman book will be the best resource for this, use it to define loss and other necessary parameters. See specific notes

Current: Find \sout{$B_r$, $K_a$}, $\mu_{rn}$, $K_{rn}$, $K_{\theta{}n}$ and coefficients for correction factor

Ask Dustin about the permanent magnets, and their magnetic remanence

The magnets are (for now) N48H neodymium.

\sout{Very important: Need the magnetization profile of the halbach array to determine  $K_{rn}$, $K_{\theta{}n}$. Will temporarily assume radial orientation.}

Very important: Need to compute fourier series of slot correction factor function (piecewise)
Need to debug, values explode with higher truncation point

\sout{Still an issue. Finish re-derivation, give it another shot for a day, then try an FEM approach.}
Justin helped me solve it by solving the linear system for every $n$. Now check the answer and see what I can do with it.

\sout{Implement slot correction factor. Clean up code.}

Implement core loss equation. Consider options for determining coefficients.

\section{High-Level Notes}

Goal: Make strides in unifying motor design process using OpenMDAO.

Ideas/paths for making progress 
\begin{itemize}
	\item Implement the lowest-fidelity analyses possible (existing surrogate models, analytic stuff)
	\item Focus on improving fidelity of one, two, analyses (EM, Thermal)
	\item Focus on defining optimization goals (objectives, constraints, design space)
  \item Menial tasks: analytic derivatives, clean up code, split across .py files
\end{itemize}

1. 
Seems to be current focus.
Going to look through MotorCAD manual to replicate their models with Aaron

Communicating with Tucker, will send some material on EM when he can

Also need to consider thermal and rotordynamic analyses

2.
Build something in EMpy?
Or MFEM?

Haven't started on this yet


3.
With losses modeled as they are (dependent only on electric frequency and representing a different motor), can't use them as objective.
Want to try minimizing mass instead, see what optimizer does.

Currently using design space bounds,

Upper limit on current density 

Lower limit on torque 

Objective is losses, but don't make sense to use as they are

4. 
Do these to pass the time

Especially need to clean up comments, remove deprecated lines




\section{Notable Commits}

ef18621 2019-07-02: Added this notes.tex file

91029b2 2019-07-02: The surrogate models for magnetic/wiring eddy and core losses work, as do the mechanical loss models. Optimization doesn't make sense as they are.

bb0d8cd 2019-07-23: Computed magnetic flux density in the stator teeth, expandable to yoke.

\section{Specific Notes}

May need to look at magnetic flux density/max flux calculations (assumed constant) See pg. 156 EMF book

Model magnetic field as a function of the properties of the coils in the motor. See pg. 161 EMF book

MATlab code modeling three phase rotating magnetic field. See pg. 168 EMF book

Core loss reductions, high dependence on lamination thickness and resisitivity of material (find this out for Hiperco). See pg. 219 Hanselman

Amazon links to good books:

https://www.amazon.com/Electric-Machinery-Fundamentals-Chapman/dp/007108617X

Hysteresis Losses (Steinmetz Equations):

\begin{equation}
P_h = k_hfB_{pk}^n
\end{equation}

or 

\begin{equation}
P_h = k_hfB_{pk}^{n+mB_{pk}}
\end{equation}

$k_h$, $n$, $m$, are material constants. $B_{pk}$ is peak flux density,

To be useful for optimization, we need to compute $B$ in general (steady state)

For $B_{pk}$,

\begin{equation}
B(\theta) = \sum_{n=-\infty}^{\infty}B_ne^{jn\theta}
\end{equation}

And $B_o={pk}$ is maximum of $B(\theta)$. Check (7.23) and (7,24) for stator tooth, (7.32) and (7.33) for yoke. (pg. 158 Hanselman)

\begin{equation}
\phi_i(\alpha) = \sum_{n=-\infty}^{\infty}\bigg\{\frac{2L_{st}R_s}{N_m}B_{gn}\int_{-\theta_s/2}^{\theta_s/2}K_{sl}(\theta)e^{jn\theta}d\theta{}\bigg\}e^{jn\alpha}
\end{equation}

But I need to see if the magnetic field is sinusoidal.

Halbach array is said to have sinusoidal magnetic field profile.???

Model appears to suffer from subtractive cancellation due to exponential increase with $n$. Higher frequency terms destroy accuracy. 



Caveats to keep in mind:

Model suffers from subtractive cancellation error for large $\beta$, limiting accuracy of solution. Should not severely impact solution.

Model depends on knowledge of the magnetic field profile. Unsure how to characterize Halbach array, but some options are implemented.

Need to derive material constants for Hiperco 50 so as to fit equation 9.39, then use 9.40 to generally compute loss. Material data sheets may have the data to do this, but otherwise may require experimentation. Current numbers are a guess.

Model currently only computes core losses only within the stator teeth, but we can now quickly extend to other relevant losses.


\end{document}
